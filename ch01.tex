\chapter{Discrete Structures Intro Statements}
\dfn{Statements}{Declarative sentence with truth value 
		\begin{itemize}
		  \item Non subjective, contains defined true or             false (Never both)
            \item Logical, non meaningless. Statements mean something
            \item Does not need to be verifiable:  Ex.\begin{itemize}
                \item There is exactly 12 people sneezing at this second
                \item Aliens exist
            \end{itemize}
		\end{itemize}}
\ex{Statement Examples}{\begin{itemize}
    \item $2+2=4$
    \item $2+2=0$
    \item The sky is blue
\end{itemize}}
Note above how both $2+2=4$ and $2+2=0$ are both valid statements despite them contradicting each other. Both have a defined true-false value and thus can both be perfectly valid as a statement. $2+2=4$ is always true, and $2+2=0$ is always false, no question about it.
\ex{Non-Statement Examples}{\begin{itemize}
    \item $x>30$
    \item CMSC250 is hard
\end{itemize}}
The first line is not a statement as $x>30$ is subjective. What is x? if x is 40 then the statement is true, but if x is 0 then the statement is false. Thus there is no defined true-false value. Until x is defined, it is not a statement.

Similarly the second non-statement is false. According to what metric is CMSC250 hard? Some people will agree or disagree, making it subjective and thus not a statement. We can easily fix the two above into statements however with a few quick word changes
\ex{Fixed Statements}{\begin{itemize}
    \item $x=50: x>30$
    \item I think CMSC250 is hard
\end{itemize}}
These are now both valid statements, as with a defined x value, we know x will always be greater than 30.
Similarly we defined the second statement by adjusting it from a vague subjective thought, to a statement of my opinion. A statement of my opinion is always true or false.
\section{Statement Notation}
Statements are notated at all times with a lowercase letter, I.E.
$s=$ the sky is blue. Additionally there are various special symbols that can be used to modify statements and their relation to other statements
\dfn{Statement Symbols}{
\begin{itemize}
    \item $\neg$ symbolizes Negation (Not)
    \item $\wedge$ symbolizes Conjunction (And)
    \item $\vee$ symbolizes Disjunction (Or)
    \item $\oplus$ symbolizes mutually exclusive and/or (XOR)
    \begin{itemize}
        \item *Note: We will not use XOR as it can be expressed with a combination of conjunctions $\wedge$ and disjunctions $\vee$
        \item Ex: $(p{\wedge}{\neg}q) {\vee} ({\neg}p{\wedge}q)$
        \item p and not q or not p and q 
    \end{itemize}
\end{itemize}
}
\ex{Statement Examples}{Take these two statements:

$s=$French fries are green

$q=$The sky is blue\begin{itemize}
\item{Conjunction Example: s $\wedge$ q - False}
\item{Disjunction Example: s $\vee$ q - True}
\item{Not example: $\neg$ s - True}
\end{itemize}}
Like more traditional math, Statement Symbols have a defined order of operations in a sense that notation matters. While read left to right, without () we do not have defined orders. However $\neg$ always takes priority
\qs{Statement Ordering}{Notate the following: Today I have a discussion or lecture and a presentation\begin{itemize}
    \item p: I have a discussion
    \item q: I have a lecture
    \item r: I have a presentation
\end{itemize}
Now consider two possible notations: 
\begin{itemize}
    \item (p $\vee$ q) $\wedge$ r
    \item p $\vee$ (q $\wedge$ r)
\end{itemize}
Both can be true, so we must use () to dictate what we mean}