\chapter{Normed Linear Space}
\dfn{Limit of Sequence in $\bs{\bbR}$}{Let $\{s_n\}$ be a sequence in $\bbR$. We say $$\lim_{n\to\infty}s_n=s$$ where $s\in\bbR$ if $\forall$ real numbers $\eps>0$ $\exists$ natural number $N$ such that for $n>N$ $$s-\eps<s_n<s+\eps\text{ i.e. }|s-s_n|<\eps$$}

Want to generalize this to a sequence in $\bbR^n$ i.e. $s_n\in\bbR^n$ $\forall$ $n\in\bbN$. Now the $s-s_n$ makes no sense. So it is useful to have a notion of whether vectors are big or small. We have magnitude of a vector. So lets revisit this

\dfn{Limit of Sequence in $\bs{\bbR^n}$}{Let $\{s_n\}$ be a sequence in $\bbR^n$. We say $$\lim_{n\to\infty}s_n=s$$ where $s\in\bbR^n$ if $\forall$ real numbers $\eps>0$ $\exists$ natural number $N$ such that for $n>N$ $$\|s-s_n\|<\eps$$}
The same definition works if we interpret $\|v\|=$ length of the vector $v$.

From school, for v=$v_1,v_2,\cdots,v_n$ we had $$\|v\|=\sqrt{v_1^2+v_2^2+\cdots+v_n^2}$$But it will be useful to have a more general notion of length (of which the above will be an example)
\section{Defination}
\dfn{\label{norm}Normed Linear Space and Norm $\boldsymbol{\|\cdot\|}$}{Let $V$ be a vector space over $\bbR$ (or $\bbC$). A norm on $V$ is function $\|\cdot\|\ V\to \bbR_{\geq 0}$ satisfying \begin{enumerate}[label=\bfseries\tiny\protect\circled{\small\arabic*}]
		\item \label{n:1}$\|x\|=0 \iff x=0$ $\forall$ $x\in V$
		\item \label{n:2}	$\|\lambda x\|=|\lambda|\|x\|$ $\forall$ $\lambda\in\bbR$(or $\bbC$), $x\in V$
		\item \label{n:3} $\|x+y\| \leq \|x\|+\|y\|$ $\forall$ $x,y\in V$ (Triangle Inequality/Subadditivity)
	\end{enumerate}And $V$ is called a normed linear space.

	$\bullet $ Same definition works with $V$ a vector space over $\bbC$ (again $\|\cdot\|\to\bbR_{\geq 0}$) where \ref{n:2} becomes $\|\lambda x\|=|\lambda|\|x\|$ $\forall$ $\lambda\in\bbC$, $x\in V$, where for $\lambda=a+ib$, $|\lambda|=\sqrt{a^2+b^2}$ }


\ex{$\bs{p-}$Norm}{\label{pnorm}$V={\bbR}^m$, $p\in\bbR_{\geq 0}$. Define for $x=(x_1,x_2,\cdots,x_m)\in\bbR^m$ $$\|x\|_p=\Big(|x_1|^p+|x_2|^p+\cdots+|x_m|^p\Big)^{\frac1p}$$(In school $p=2$)}
\textbf{Special Case $\bs{p=1}$}: $\|x\|_1=|x_1|+|x_2|+\cdots+|x_m|$ is clearly a norm by usual triangle inequality. \par
\textbf{Special Case $\bs{p\to\infty\ (\bbR^m$ with $\|\cdot\|_{\infty})}$}: $\|x\|_{\infty}=\max\{|x_1|,|x_2|,\cdots,|x_m|\}$\\
For $m=1$ these $p-$norms are nothing but $|x|$.
Now exercise
\qs{}{\label{exs1}Prove that triangle inequality is true if $p\geq 1$ for $p-$norms. (What goes wrong for $p<1$ ?)}
\sol{\textbf{For Property \ref{n:3} for norm-2}	\subsubsection*{\textbf{When field is $\bbR:$}} We have to show\begin{align*}
		         & \sum_i(x_i+y_i)^2\leq \left(\sqrt{\sum_ix_i^2} +\sqrt{\sum_iy_i^2}\right)^2                                       \\
		\implies & \sum_i (x_i^2+2x_iy_i+y_i^2)\leq \sum_ix_i^2+2\sqrt{\left[\sum_ix_i^2\right]\left[\sum_iy_i^2\right]}+\sum_iy_i^2 \\
		\implies & \left[\sum_ix_iy_i\right]^2\leq \left[\sum_ix_i^2\right]\left[\sum_iy_i^2\right]
	\end{align*}So in other words prove $\langle x,y\rangle^2 \leq \langle x,x\rangle\langle y,y\rangle$ where
	$$\langle x,y\rangle =\sum\limits_i x_iy_i$$

	\begin{note}
		\begin{itemize}
			\item $\|x\|^2=\langle x,x\rangle$
			\item $\langle x,y\rangle=\langle y,x\rangle$
			\item $\langle \cdot,\cdot\rangle$ is $\bbR-$linear in each slot i.e. \begin{align*}
				      \langle rx+x',y\rangle=r\langle x,y\rangle+\langle x',y\rangle	\text{ and similarly for second slot}
			      \end{align*}Here in $\langle x,y\rangle$ $x$ is in first slot and $y$ is in second slot.
		\end{itemize}
	\end{note}Now the statement is just the Cauchy-Schwartz Inequality. For proof $$\langle x,y\rangle^2\leq \langle x,x\rangle\langle y,y\rangle $$ expand everything of $\langle x-\lambda y,x-\lambda y\rangle$ which is going to give a quadratic equation in variable $\lambda $ \begin{align*}
		\langle x-\lambda y,x-\lambda y\rangle & =\langle x,x-\lambda y\rangle-\lambda\langle y,x-\lambda y\rangle                                       \\
		                                       & =\langle x ,x\rangle -\lambda\langle x,y\rangle -\lambda\langle y,x\rangle +\lambda^2\langle y,y\rangle \\
		                                       & =\langle x,x\rangle -2\lambda\langle x,y\rangle+\lambda^2\langle y,y\rangle
	\end{align*}Now unless $x=\lambda y$ we have $\langle x-\lambda y,x-\lambda y\rangle>0$ Hence the quadratic equation has no root therefore the discriminant is greater than zero.

	\subsubsection*{\textbf{When field is $\bbC:$}}Modify the definition by $$\langle x,y\rangle=\sum_i\overline{x_i}y_i$$Then we still have $\langle x,x\rangle\geq 0$}
\section{Open and Closed Ball}
\dfn{Open and Closed Ball in Normed Linear Space}{An open Ball of radius $r$ with center $x$ in Normed Linear Space $V$ is the set$$\{y\in V\mid \|x-y\|<r\}=B_r(x)$$and Closed ball is the set $$\{y\in V\mid\|x-y\|\leq r\}=\overline{B_r(x)}$$}


Now take $B_r(0)$ w.r.t \textcolor{myr}{$\bs{\|\cdot\|_1}$}, \textcolor{myg}{$\bs{\|\cdot\|_2}$}, \textcolor{myb}{$\bs{\|\cdot\|_{\infty}}$}.
%\begin{figure}[h]
%	\centering
%	\includegraphics[width=4cm]{images/1.png}
%\end{figure}
Now imagine a sequence converging to origin. So if I
\begin{center}
	\begin{tikzpicture}
		\draw (-2.5, 0) -- (2.5, 0) ;
		\draw (0, -2.5) -- (0, 2.5) ;
		\draw[myg,thick] (0,0) circle (2cm);
		\draw[myb,thick] (-2,2) -- (2,2) -- (2,-2) -- (-2,-2) -- cycle;
		\draw[myr,thick] (2,0) -- (0,-2) -- (-2,0) -- (0,2) --cycle;
	\end{tikzpicture}
\end{center}

draw an ordinary circle around the origin then no matter how small the circle the points of the sequence are eventually land inside the circle. If instead of that circle can same be said for diamond w.r.t norm 2. Then i can take circle that is inside that diamond. Same is true for $\infty-$norm. Hence convergence with respect to all norm $1$ and norm 2 and even $\infty$ results for convergence.






Now there is no reason why we can not consider a norm on an infinite dimensional vector space. It will work. Perhaps i can define only for some sequences where the morm converges. \ex{}{Suppose for set of all bounded infinite sequences a vector space because every number in a vector is less than some number so if you add two vectors then add the bound and if you scale then scale the bound. Now the $\infty$ norm works on that.

	Now suppose you take all continuous real valued functions on closed interval $[0,1]$, such a function is bounded and this is a vector space and we can define $\infty-$norm even for that because for all $f$ in this space attains its maximum value so just take that maximum value. Its an extremely infinite dimensional space.


}

\begin{note}
	$\bbR^{\infty}$ is the space of all sequences.
\end{note}

\qs{}{\label{exs2}Modify the above proof for field $\bbC$}
\qs{}{\label{exs3}Show that the following are normed linear spaces.\begin{enumerate}[label=(\alph*)]
		\item $l^{\infty}=$ Set of all bounded infinite sequences $(x_1,x_2,\cdots)$ $x_i\in\bbR$ with norm $\|x\|=\sup |x_i|$
		\item $C[0,1]=$ Set of all continuous functions $[0,1]\to \bbR$ with norm $\|f\|=\sup\limits_{x\in[0,1]}|f(x)|$
	\end{enumerate}
}
\section{Limit of a Sequence}
\dfn{Limit of Sequence in Normed Linear Space}{A sequence  $\{s_n\}$ in a normed linear space $V$ converge to $s$ means $\forall$ real number $\eps>0$ $\exists$ natural number $N$ such that for $\forall\ n>N$ $\|s-s_n\|<\eps$}
\section{Continuity}
\dfn{Continuity in Normed Linear Space}{Let $S$ be a subset of $V$ and $f:\ S\to W$ where $V,W$ are normed linear space. $f$ is continuous at $v\in V$ means $\forall$ $\eps>0$, $\exists$ $\delta>0$, st whenever $\|x-v\|<\delta$ for $x\in S$ one has $\|f(x)-f(v)\|<\eps$}
Distance in a normed linear space for $x,y\in V$ is $$d(x,y)=\|x,y\|$$ Hence properties of this $d$ are\begin{enumerate}[label=\bfseries\tiny\protect\circled{\small\arabic*}]
	\item[\ref{n:1}] \label{m:1} $d(x,y)=0 \iff x=y$
	\item[\ref{n:2}] \label{m:2} $d(\lambda x,\lambda y)=|\lambda|d(x,y)$ for any scalar $\lambda$
	\item[\ref{n:3}] \label{m:3} $d(u,v)+d(u,v)\geq d(u,w)$
\end{enumerate}
